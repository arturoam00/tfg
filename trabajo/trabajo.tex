\documentclass[11pt, a4paper]{article} %tamaño mínimo de letra 11pto.

\usepackage{graphicx} 
\usepackage[spanish]{babel} %Español 
\usepackage[utf8]{inputenc} %Para poder poner tildes
\usepackage{vmargin} %Para modificar los márgenes
\setmargins{2.5cm}{1.5cm}{16.5cm}{23.42cm}{10pt}{1cm}{0pt}{2cm}
%margen izquierdo, superior, anchura del texto, altura del texto, altura de los encabezados, espacio entre el texto y los encabezados, altura del pie de página, espacio entre el texto y el pie de página

\begin{document}
%%%%%%Portada%%%%%%%
\begin{titlepage}
\centering
{ \bfseries \Large UNIVERSIDAD COMPLUTENSE DE MADRID}
\vspace{0.5cm}

{\bfseries  \Large FACULTAD DE CIENCIAS FÍSICAS} 
\vspace{1cm}

{\large DEPARTAMENTO DE XXXXX}
\vspace{0.8cm}

%%%%Logo Complutense%%%%%
{\includegraphics[width=0.35\textwidth]{logo_UCM}} %Para ajustar la portada a una sola página se puede reducir el tamaño del logo
\vspace{0.8cm}

{\bfseries \Large TRABAJO DE FIN DE GRADO}
\vspace{2cm}

{\Large Código de TFG:  [C\'odigo TFG] } \vspace{5mm}

{\Large [Título de TFG (exactamente el que aparece en la FICHA)]}\vspace{5mm}

{\Large [Título de TFG en ingl\'es (el que aparece en la FICHA)]}\vspace{5mm}

{\Large Supervisor/es: [Nombre del/os supervisores]}\vspace{20mm} 

{\bfseries \LARGE [Nombre del alumno]}\vspace{5mm} 

{\large Grado en Física}\vspace{5mm} 

{\large Curso acad\'emico 20[XX-XX]}\vspace{5mm} 

{\large Convocatoria XXXX}\vspace{5mm} 

\end{titlepage}
\newpage

{\bfseries \large [Título extendido del TFG (si procede)] }\vspace{10mm} 

{\bfseries \large Resumen:} \vspace{5mm}

Esto es una prueba para probar el formato del Resumen. Esto es una prueba para probar el formato del ResumenEsto es una prueba para probar el formato del ResumenEsto es una prueba para probar el formato del ResumenEsto es una prueba para probar el formato del ResumenEsto es una prueba para probar el formato del ResumenEsto es una prueba para probar el formato del ResumenEsto es una prueba para probar el formato del ResumenEsto es una prueba para probar el formato del ResumenEsto es una prueba para probar el formato del ResumenEsto es una prueba para probar el formato del ResumenEsto es una prueba para probar el formato del ResumenEsto es una prueba para probar el formato del ResumenEsto es una prueba para probar el formato del ResumenEsto es una prueba para probar el formato del Resumen.
\vspace{1cm}

{\bfseries \large Abstract: }\vspace{5mm} 

This is a test to prove the abstract's layout.This is a test to prove the abstract's layout.This is a test to prove the abstract's layout.This is a test to prove the abstract's layout.This is a test to prove the abstract's layout.This is a test to prove the abstract's layout.This is a test to prove the abstract's layout.This is a test to prove the abstract's layout.This is a test to prove the abstract's layout.This is a test to prove the abstract's layout.This is a test to prove the abstract's layout.This is a test to prove the abstract's layout.This is a test to prove the abstract's layout.This is a test to prove the abstract's layout.This is a test to prove the abstract's layout.This is a test to prove the abstract's layout.This is a test to prove the abstract's layout.This is a test to prove the abstract's layout.This is a test to prove the abstract's layout.
\vspace{1cm}

[\Large\textbf{Nota}: el título extendido (si procede), el resumen y el abstract deben estar en una misma página y su extensión no debe superar una página. Tamaño mínimo 11pto ]\\

{\Large\textbf{Extensión máxima 20 páginas sin contar portada ni resumen (sí se incluye índice, introducción, conclusiones y bibliografía}}
\newpage

%%Inicio: 


\end{document}